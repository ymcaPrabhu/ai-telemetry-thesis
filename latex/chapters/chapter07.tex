\chapter{Policy, Governance, and Compliance Implications}\label{chap:policy}
\section{Alignment with Indian Policies}
The framework automates compliance with CERT-In directives by generating incident reports within mandated timelines, storing logs for 180 days, and preserving audit integrity. It supports the National Cyber Security Strategy by enabling threat intelligence sharing and capacity building.

\section{Cloud Standards}
Mappings to ISO/IEC~27017/27018, CSA Cloud Controls Matrix, and RBI cloud circulars confirm adherence to security best practices. Policy-as-code modules enforce data residency, encryption, and access control across AWS, Azure, and Google Cloud.

\section{Ethical AI}
Explainable AI modules comply with EU AI Act transparency obligations and mitigate bias through fairness monitoring. Model cards document intended use, limitations, and mitigation strategies.

\section{Operationalization Roadmap}
\begin{itemize}
    \item People: Train SOC teams and align roles with NIST NICE framework.
    \item Process: Integrate detections into incident response plans and establish feedback loops.
    \item Technology: Deploy via DevSecOps pipelines and integrate with existing SIEM/SOAR.
    \item Metrics: Track MTTD, MTTR, compliance KPIs, and analyst satisfaction.
\end{itemize}

\section{Cross-Jurisdictional Compliance}
Guidelines for organizations operating across India, EU, US, and ASEAN outline nuanced requirements for data residency, breach notification, and retention. Configurable policies adapt to jurisdictional variations.

\section{Cost-Benefit and Sustainability}
Cost analysis indicates moderate cloud expenditure offset by reduced breach risk and compliance penalties. Sustainability measures include carbon-aware scheduling and resource optimization.

\section{Policy Recommendations}
The thesis recommends adopting AI-assisted detection for critical infrastructure, standardizing telemetry formats for public-private collaboration, and promoting responsible AI certification for security tools.
