\chapter{Conclusion and Future Work}\label{chap:conclusion}
\section{Summary of Contributions}
The thesis introduced an AI-driven multi-telemetry framework that harmonizes heterogeneous cloud telemetry, applies hybrid AI ensembles, integrates explainability and automation, and aligns with regulatory mandates. Empirical evaluation demonstrated significant improvements in detection accuracy, latency, and analyst trust.

\section{Theoretical and Practical Implications}
The research advances multi-telemetry fusion, hybrid AI reasoning, and policy-aware detection, contributing to design science literature. Practitioners gain a deployable blueprint, playbooks, and compliance mappings for cloud security operations.

\section{Limitations}
Limitations include reliance on lab-generated telemetry, limited real-world breach data, and modest analyst sample sizes in user studies.

\section{Future Research}
Future work should explore real-world deployments, federated learning and privacy-preserving analytics, adaptive defenses via reinforcement learning, standardized telemetry ontologies, and integration with quantum-safe cryptographic controls.

\section{Closing Remarks}
By harmonizing AI, multi-telemetry analytics, and governance, the framework contributes to resilient cloud security. Collaboration among academia, industry, and government remains essential to counter evolving threats.
