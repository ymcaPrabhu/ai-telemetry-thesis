\chapter{Experimental Evaluation and Results}\label{chap:eval}

\section{Introduction}\label{sec:eval-intro}
This chapter reports comprehensive evaluation of the AI-driven multi-telemetry framework across quantitative benchmarks, qualitative analyst studies, and operational metrics. Section~\ref{sec:eval-datasets} describes evaluation datasets; Section~\ref{sec:eval-baseline} compares performance against baselines; Section~\ref{sec:eval-ablation} presents ablation studies; Section~\ref{sec:eval-scenarios} analyzes attack scenario detection; Section~\ref{sec:eval-xai} evaluates explainability and analyst feedback; Section~\ref{sec:eval-fairness} examines fairness and bias; Section~\ref{sec:eval-performance} reports operational performance metrics; Section~\ref{sec:eval-compliance} validates compliance readiness; and Section~\ref{sec:eval-discussion} discusses findings in relation to research hypotheses.

\section{Evaluation Datasets}\label{sec:eval-datasets}
\subsection{Benchmark Datasets}
\textbf{CIC-IDS2017:}
\begin{itemize}
    \item \textbf{Composition:} 2.8M network flow records (80\% benign, 20\% attacks)
    \item \textbf{Attack Types:} Brute force, DoS, DDoS, web attacks, infiltration, port scan, botnet
    \item \textbf{Usage:} Baseline model training and comparison
    \item \textbf{Preprocessing:} 83 features extracted via CICFlowMeter; normalized; 60/20/20 train/val/test split
\end{itemize}

\textbf{UNSW-NB15:}
\begin{itemize}
    \item \textbf{Composition:} 2.5M records across 9 attack categories
    \item \textbf{Features:} 49 network flow features
    \item \textbf{Usage:} Generalization testing across diverse attack patterns
\end{itemize}

\textbf{DARPA OpTC:}
\begin{itemize}
    \item \textbf{Composition:} System provenance graphs from 5 APT campaigns
    \item \textbf{Scale:} 1.2B process/file/network events over 30 days
    \item \textbf{Usage:} Multi-stage attack detection, graph-based reasoning validation
\end{itemize}

\subsection{Custom Cloud Telemetry Corpus}
Collected over 90-day period from sandboxed cloud environments:

\textbf{AWS Telemetry (485 GB):}
\begin{itemize}
    \item CloudTrail: 127M API calls
    \item VPC Flow Logs: 2.8B network flows
    \item Lambda logs: 45M function invocations
    \item S3 access logs: 89M requests
    \item GuardDuty findings: 1,247 alerts (42 true positives, 1,205 false positives)
\end{itemize}

\textbf{Azure Telemetry (312 GB):}
\begin{itemize}
    \item Activity Logs: 78M management operations
    \item NSG Flow Logs: 1.9B flows
    \item Azure AD Sign-ins: 12M authentication events
    \item Application Insights: 34M distributed traces
\end{itemize}

\textbf{Google Cloud Telemetry (267 GB):}
\begin{itemize}
    \item Cloud Audit Logs: 91M operations
    \item VPC Flow Logs: 1.6B flows
    \item Cloud Functions logs: 28M invocations
    \item Security Command Center: 843 findings
\end{itemize}

\textbf{Attack Scenarios:} Five multi-stage campaigns (S1-S5) executed 10 times each:
\begin{itemize}
    \item S1 (Credential Chain): 50 attack instances, avg 8.3 steps, duration 12-47 minutes
    \item S2 (Lateral Movement): 50 instances, avg 11.7 steps, duration 18-62 minutes
    \item S3 (Serverless Exfiltration): 50 instances, avg 6.2 steps, duration 8-24 minutes
    \item S4 (Cryptomining): 50 instances, avg 4.8 steps, duration 5-18 minutes
    \item S5 (SaaS Supply Chain): 50 instances, avg 9.5 steps, duration 15-53 minutes
\end{itemize}

Each scenario executed 10 times with timing variations to assess detection robustness.

The use of a diverse range of datasets is critical for a comprehensive evaluation of the framework. The benchmark datasets provide a baseline for comparison with other research, while the custom cloud telemetry corpus allows for a more realistic evaluation of the framework's performance in a real-world setting~\cite{irejournals2024datasets}.

The evaluation of AI-driven security frameworks is a complex and challenging task. It requires a combination of robust metrics, real-world case studies, and a deep understanding of the underlying technology. This research will employ a multi-faceted evaluation strategy to provide a comprehensive assessment of the framework's performance~\cite{jsaer2024evaluation}.

\section{Baseline Comparison}\label{sec:eval-baseline}
\subsection{Detection Performance Metrics}
Table~\ref{tab:baseline-comparison} summarizes performance across methods:

\begin{table}[H]
\centering
\caption{Detection performance comparison (CIC-IDS2017 test set)}
\label{tab:baseline-comparison}
\begin{tabular}{lcccccc}
\toprule
\textbf{Method} & \textbf{Precision} & \textbf{Recall} & \textbf{F1} & \textbf{ROC-AUC} & \textbf{FPR (\%)} & \textbf{Latency (ms)} \\
\midrule
Sigma Rules (SIEM) & 0.78 & 0.71 & 0.74 & 0.82 & 6.2 & 25 \\
Random Forest & 0.82 & 0.79 & 0.80 & 0.88 & 4.9 & 12 \\
Isolation Forest & 0.71 & 0.84 & 0.77 & 0.85 & 8.7 & 8 \\
LSTM & 0.86 & 0.81 & 0.83 & 0.90 & 3.8 & 48 \\
GNN (GAT only) & 0.88 & 0.85 & 0.86 & 0.92 & 3.2 & 67 \\
TCN only & 0.87 & 0.83 & 0.85 & 0.91 & 3.5 & 52 \\
\midrule
\textbf{Proposed Ensemble} & \textbf{0.93} & \textbf{0.90} & \textbf{0.91} & \textbf{0.96} & \textbf{2.4} & \textbf{82} \\
\bottomrule
\end{tabular}
\end{table}

\textbf{Statistical Significance:}
Paired t-tests comparing ensemble vs. each baseline (n=250 runs, 5-fold CV, 50 repetitions):
\begin{itemize}
    \item Ensemble vs. Sigma Rules: $t(249) = 18.3, p < 0.001$
    \item Ensemble vs. Random Forest: $t(249) = 12.7, p < 0.001$
    \item Ensemble vs. LSTM: $t(249) = 8.4, p < 0.001$
    \item Ensemble vs. GAT only: $t(249) = 5.2, p < 0.001$
\end{itemize}

All comparisons significant after Bonferroni correction ($\alpha = 0.05/6 = 0.0083$).

The limitations of traditional signature-based methods are well-documented. These methods are often unable to detect novel and sophisticated attacks, and they can be difficult to maintain in a dynamic cloud environment. This research will demonstrate the superiority of an AI-driven approach to cloud security, which is able to learn and adapt to new threats in real-time~\cite{journalwjarr2024baselines}.

The evaluation of AI-driven security frameworks requires a comprehensive set of metrics that go beyond traditional measures of accuracy. This research will use a variety of metrics to evaluate the performance of the framework, including the false positive rate, the mean time to detect, and the mean time to respond. This will provide a more complete picture of the framework's effectiveness in a real-world setting~\cite{aisnet2024evaluation}.

\subsection{Cloud-Specific Performance}
Table~\ref{tab:cloud-performance} shows detection performance on custom cloud telemetry:

\begin{table}[H]
\centering
\caption{Detection performance on cloud attack scenarios (S1-S5)}
\label{tab:cloud-performance}
\begin{tabular}{lccccc}
\toprule
\textbf{Scenario} & \textbf{Detected} & \textbf{Missed} & \textbf{Detection Rate} & \textbf{Avg Latency (s)} & \textbf{FP} \\
\midrule
S1: Credential Chain & 47/50 & 3 & 94\% & 42 & 2 \\
S2: Lateral Movement & 48/50 & 2 & 96\% & 58 & 1 \\
S3: Serverless Exfil & 44/50 & 6 & 88\% & 35 & 4 \\
S4: Cryptomining & 50/50 & 0 & 100\% & 28 & 0 \\
S5: SaaS Supply Chain & 46/50 & 4 & 92\% & 67 & 3 \\
\midrule
\textbf{Overall} & \textbf{235/250} & \textbf{15} & \textbf{94\%} & \textbf{46} & \textbf{10} \\
\bottomrule
\end{tabular}
\end{table}

\textbf{False Negatives Analysis:}
\begin{itemize}
    \item 8 cases: Low-volume exfiltration below threshold (< 10 MB)
    \item 4 cases: Delayed telemetry arrival (S3 access logs batched)
    \item 3 cases: Novel techniques not in training data (zero-day simulation)
\end{itemize}

\section{Ablation Studies}\label{sec:eval-ablation}
\subsection{Component Contribution Analysis}
Table~\ref{tab:ablation} isolates individual component contributions:

\begin{table}[H]
\centering
\caption{Ablation study results (cloud telemetry corpus)}
\label{tab:ablation}
\begin{tabular}{lcccc}
\toprule
\textbf{Configuration} & \textbf{Precision} & \textbf{Recall} & \textbf{F1} & \textbf{$\Delta$ F1 from Full} \\
\midrule
Full Ensemble (GAT+TCN+BBN) & 0.93 & 0.90 & 0.91 & --- \\
Without GAT (TCN+BBN) & 0.89 & 0.86 & 0.87 & -0.04 \\
Without TCN (GAT+BBN) & 0.90 & 0.84 & 0.87 & -0.04 \\
Without BBN (GAT+TCN) & 0.91 & 0.88 & 0.89 & -0.02 \\
GAT only & 0.88 & 0.85 & 0.86 & -0.05 \\
TCN only & 0.87 & 0.83 & 0.85 & -0.06 \\
BBN only & 0.79 & 0.74 & 0.76 & -0.15 \\
\bottomrule
\end{tabular}
\end{table}

\textbf{Findings:}
\begin{itemize}
    \item GAT critical for detecting lateral movement and privilege escalation (entity relationship reasoning)
    \item TCN essential for temporal attack sequences (delayed exfiltration, periodic C2 beaconing)
    \item BBN provides probabilistic aggregation but lower standalone performance
    \item Ensemble synergy: Combined F1 exceeds best individual model by 0.05
\end{itemize}

\subsection{Telemetry Source Contribution}
Evaluated detection performance with subsets of telemetry sources:

\begin{table}[H]
\centering
\caption{Telemetry source ablation (S1-S5 scenarios)}
\label{tab:telemetry-ablation}
\begin{tabular}{lcc}
\toprule
\textbf{Telemetry Sources} & \textbf{Detection Rate} & \textbf{Avg Latency (s)} \\
\midrule
All sources (network + identity + compute + storage) & 94\% & 46 \\
Network + identity + compute only & 89\% & 52 \\
Network + identity only & 78\% & 68 \\
Network only & 62\% & 81 \\
Identity only & 71\% & 43 \\
\bottomrule
\end{tabular}
\end{table}

Conclusion: Multi-telemetry fusion critical; single-source detection misses 38\% of attacks. Identity telemetry alone detects 71\%, indicating IAM abuse prevalence.

Ablation studies are a key component of the evaluation strategy. By systematically removing components from the framework and measuring the impact on performance, it is possible to gain a deeper understanding of the contribution of each component. This can help to identify the most important features of the framework and to guide future research and development~\cite{researchgate2024ablation}.

The use of ablation studies in deep learning for cybersecurity is a relatively new but growing area of research. These studies can provide valuable insights into the inner workings of complex models and can help to improve their performance and interpretability~\cite{mdpi2024ablation}.

\section{Attack Scenario Deep Dives}\label{sec:eval-scenarios}
\subsection{S1: Credential Compromise Chain}
\textbf{Attack Narrative:} Attacker phishes credentials, bypasses MFA via session hijacking, assumes privileged role, escalates to admin, accesses sensitive data.

\textbf{Detection:}
\begin{itemize}
    \item \textbf{Stage 1 (Phishing):} Anomalous login location detected by GAT (unusual geo-jump)
    \item \textbf{Stage 2 (MFA Bypass):} TCN identified session token reuse pattern
    \item \textbf{Stage 3 (Role Assumption):} BBN flagged AssumeRole chain without MFA re-auth
    \item \textbf{Stage 4 (Privilege Escalation):} GAT detected IAM policy attachment to self
    \item \textbf{Stage 5 (Data Access):} S3 GetObject on high-sensitivity bucket triggered alert
\end{itemize}

\textbf{Alert Generated:} 42 seconds after initial anomalous login (Stage 1). Consolidated alert covered full attack chain.

\textbf{SHAP Explanation:} Top features: unusual login region (0.31), rapid role assumption (0.28), IAM policy change (0.24), S3 bucket sensitivity (0.17).

\subsection{S2: Cross-Account Lateral Movement}
\textbf{Detection:} GAT identified cross-account trust relationship exploitation. Attack graph showed: Compromised account A $\rightarrow$ AssumeRole in account B $\rightarrow$ Launch EC2 $\rightarrow$ Access RDS database $\rightarrow$ Egress to external IP.

\textbf{Alert Latency:} 58 seconds. TCN captured temporal sequence of API calls. BBN computed $P(\text{lateral movement}) = 0.87$.

\subsection{S3: Serverless Exfiltration}
\textbf{Challenge:} Ephemeral Lambda functions, sparse logs.

\textbf{Detection:} TCN modeled Lambda invocation patterns. Detected anomalies: unusual S3 read volume, external network destination, rapid function invocations.

\textbf{Result:} 44/50 instances detected. 6 false negatives due to low-volume exfil below threshold.

\section{Explainability and Analyst Feedback}\label{sec:eval-xai}
\subsection{Study Protocol}
15 security analysts (mean experience: 8.4 years, SD: 3.1) completed 90-minute usability study:
\begin{enumerate}
    \item Pre-study survey: Demographics, XAI familiarity
    \item Training: Framework overview, explanation interpretation (30 min)
    \item Tasks: Investigate 8 alerts (4 TP, 4 FP) using dashboard
    \item Post-task surveys: SUS, NASA-TLX, trust questionnaire
    \item Interview: Open-ended feedback (15 min)
\end{enumerate}

\subsection{Quantitative Results}
\textbf{System Usability Scale (SUS):}
\begin{itemize}
    \item Mean SUS score: 82.3 (SD: 7.6)
    \item Interpretation: Excellent usability (threshold: 68)
    \item Comparison: Baseline SIEM dashboard scored 61.2
\end{itemize}

\textbf{NASA Task Load Index (TLX):}
\begin{itemize}
    \item Overall workload: 42.1/100 (lower is better)
    \item Mental demand: 48.3 (moderate)
    \item Performance: 78.4 (high perceived performance)
    \item Frustration: 26.7 (low)
\end{itemize}

\textbf{Trust Questionnaire:}
\begin{itemize}
    \item Trust in AI recommendations: 4.2/5.0 (SD: 0.6)
    \item Explanation quality: 4.4/5.0 (SD: 0.5)
    \item Willingness to rely on system: 4.1/5.0 (SD: 0.7)
    \item Comparison: Baseline system scored 3.1/5.0 on trust
\end{itemize}

\subsection{Qualitative Findings}
Thematic analysis identified five key themes (Cohen's kappa inter-rater reliability: 0.87):

\textbf{1. Explanation Comprehensiveness (11/15 participants):}
"SHAP values helped me understand *why* the alert fired. I could see that the unusual login location and rapid role changes were the smoking guns."

\textbf{2. Attack Narrative Utility (13/15):}
"The narrative summary was like having a senior analyst walk me through the attack. I could immediately see the full kill chain without digging through raw logs."

\textbf{3. Counterfactual Guidance (9/15):}
"Counterfactuals helped me think about legitimate exceptions. If the user had MFA, this wouldn't have triggered—good insight for tuning."

\textbf{4. Information Overload Risk (6/15):}
"Sometimes too many SHAP features. I'd prefer top 5 instead of top 20."

\textbf{5. Workflow Integration (12/15):}
"One-click SOAR actions saved time. Instead of manual containment, I could revoke credentials directly from the alert."

\subsection{Triage Performance}
Measured time-to-decision and accuracy:

\begin{table}[H]
\centering
\caption{Analyst triage performance comparison}
\label{tab:triage-performance}
\begin{tabular}{lccc}
\toprule
\textbf{Metric} & \textbf{Baseline SIEM} & \textbf{Proposed Framework} & \textbf{Improvement} \\
\midrule
Avg time-to-decision (min) & 8.7 & 5.2 & -40\% \\
Decision accuracy (\%) & 79.2 & 91.7 & +12.5 pp \\
False positive identification & 68\% & 88\% & +20 pp \\
True positive confidence (1-5) & 3.4 & 4.3 & +26\% \\
\bottomrule
\end{tabular}
\end{table}

\textbf{Statistical Analysis:} Paired t-test on time-to-decision: $t(14) = 6.8, p < 0.001$. Analysts triaged alerts 40\% faster with proposed framework.

\section{Fairness and Bias Analysis}\label{sec:eval-fairness}
\subsection{Performance Across Tenant Types}
Evaluated detection consistency across organizational profiles:

\begin{table}[H]
\centering
\caption{Detection performance by tenant type}
\label{tab:fairness-tenant}
\begin{tabular}{lccc}
\toprule
\textbf{Tenant Type} & \textbf{Precision} & \textbf{Recall} & \textbf{F1} \\
\midrule
Enterprise (>10K employees) & 0.94 & 0.91 & 0.92 \\
Mid-market (1K-10K) & 0.93 & 0.89 & 0.91 \\
SMB (<1K) & 0.91 & 0.88 & 0.89 \\
\midrule
Variance & 0.0015 & 0.0013 & 0.0014 \\
\bottomrule
\end{tabular}
\end{table}

\textbf{Conclusion:} Performance variance $<$3\% across tenant sizes. No significant bias favoring larger organizations.

\subsection{Geographical Fairness}
Detection performance by primary data center region:

\begin{table}[H]
\centering
\caption{Detection performance by region}
\label{tab:fairness-region}
\begin{tabular}{lccc}
\toprule
\textbf{Region} & \textbf{Precision} & \textbf{Recall} & \textbf{F1} \\
\midrule
North America & 0.93 & 0.90 & 0.91 \\
Europe & 0.92 & 0.91 & 0.91 \\
Asia-Pacific & 0.92 & 0.88 & 0.90 \\
\bottomrule
\end{tabular}
\end{table}

\textbf{Fairness Metric:} Demographic parity difference $<$0.02 across regions. No evidence of systematic regional bias.

\section{Operational Performance}\label{sec:eval-performance}
\subsection{Throughput and Latency}
Production deployment metrics (30-day observation period):

\begin{itemize}
    \item \textbf{Ingestion Rate:} 127K events/sec sustained (peak: 312K)
    \item \textbf{Processing Latency:} p50: 0.8s, p95: 2.3s, p99: 4.7s (ingestion to alert)
    \item \textbf{Inference Latency:} p50: 45ms, p95: 120ms, p99: 280ms (per event)
    \item \textbf{Alert Rate:} 47 alerts/day (baseline SIEM: 312 alerts/day, 85\% reduction)
    \item \textbf{True Positive Rate:} 68\% (baseline: 22\%, 3x improvement)
\end{itemize}

\subsection{Resource Utilization}
\textbf{Training Costs:}
\begin{itemize}
    \item 4x NVIDIA T4 GPUs, 6 hours per dataset
    \item AWS cost: \$127 per training run
    \item Total experimentation cost (150 runs): \$19,050
\end{itemize}

\textbf{Inference Costs (monthly):}
\begin{itemize}
    \item Compute (Kubernetes): \$2,340
    \item Storage (S3/Iceberg): \$890
    \item Data transfer: \$450
    \item \textbf{Total:} \$3,680/month for 100M events/day
\end{itemize}

\textbf{Baseline SIEM Comparison:} Splunk Cloud at equivalent volume: \$12,500/month. Proposed framework: 71\% cost reduction.

\subsection{Scalability}
Load testing with Locust:
\begin{itemize}
    \item Linear scaling up to 8M events/min
    \item Auto-scaling maintained latency SLO (p95 $<$ 3s) up to 15M events/min
    \item Bottleneck: Neo4j graph database writes at 20M events/min
\end{itemize}

\section{Compliance Validation}\label{sec:eval-compliance}
\subsection{CERT-In Directives}
\begin{itemize}
    \item \textbf{6-hour Incident Reporting:} Automated incident ticket creation within 4.2 hours (avg) of detection
    \item \textbf{180-day Log Retention:} Lakehouse tiered storage verified compliant
    \item \textbf{Service Provider Registration:} Metadata tracking validated
\end{itemize}

\subsection{GDPR Compliance}
\begin{itemize}
    \item \textbf{72-hour Breach Notification:} Alert-to-notification workflow tested at 8 hours (manual approval included)
    \item \textbf{Data Minimization:} Telemetry pseudonymization validated
    \item \textbf{Right to Erasure:} Automated data deletion upon request (tested with synthetic identities)
\end{itemize}

\subsection{Audit Trail Completeness}
\begin{itemize}
    \item \textbf{Detection Provenance:} 100\% of alerts traced to source telemetry, model version, inference parameters
    \item \textbf{Response Provenance:} Automated actions logged with timestamps, actor, approvals
    \item \textbf{Data Lineage:} End-to-end tracking from raw logs to predictions validated
\end{itemize}

\section{Discussion}\label{sec:eval-discussion}
\subsection{Hypothesis Validation}
\textbf{H1 (Telemetry Normalization):} Unified model achieved 92\% schema coverage (target: $\geq$90\%). Normalization latency: 0.8s p50 (target: $<$1s). \textbf{Supported.}

\textbf{H2 (AI Ensemble Efficacy):} Ensemble achieved 93\% precision, 90\% recall (targets: $\geq$95\%, $\geq$90\%). F1 exceeded baselines by 12\%. Analyst agreement with SHAP: 78\% (target: $\geq$75\%). \textbf{Largely supported; precision slightly below target.}

\textbf{H3 (Detection Performance):} F1 improvement: 12\% (target: $\geq$10\%). Latency: 46s avg (target: $\leq$60s). FP reduction: 51\% vs. SIEM (target: $\geq$25\%). ATT\&CK coverage: 83\% (target: $\geq$80\%). \textbf{Supported.}

\textbf{H4 (Compliance Integration):} Policy-as-code overhead: 3.2\% (target: $\leq$5\%). Compliance validation accuracy: 97\% (target: $\geq$95\%). Audit trail completeness: 100\%. \textbf{Supported.}

\subsection{Comparison to State-of-the-Art}
Literature benchmarks:
\begin{itemize}
    \item Commercial SIEM (Splunk): F1 = 0.74-0.81 (our baseline: 0.74)
    \item Academic GNN-IDS (Wang et al., 2020): F1 = 0.88 (network only)
    \item Academic XAI-SIEM (Li et al., 2021): Analyst trust = 3.5/5.0
    \item \textbf{Our framework:} F1 = 0.91, analyst trust = 4.2/5.0
\end{itemize}

\subsection{Limitations}
\begin{itemize}
    \item \textbf{Lab Telemetry:} Simulated attacks may not capture full adversary sophistication
    \item \textbf{Dataset Bias:} Training data skewed toward known attack patterns
    \item \textbf{Analyst Sample:} n=15 limits generalizability; future larger-scale studies needed
    \item \textbf{Temporal Scope:} 90-day evaluation may miss seasonal/long-term patterns
\end{itemize}

\section{Summary}
Evaluation demonstrates that the proposed framework achieves statistically significant improvements over baselines across detection accuracy (12\% F1 gain), analyst productivity (40\% faster triage), and cost efficiency (71\% reduction). Hypotheses H1, H3, and H4 are fully supported; H2 is largely supported with precision slightly below target. Qualitative feedback indicates high usability (SUS: 82) and trust (4.2/5.0). Limitations include reliance on simulated attacks and modest analyst sample size. Chapter~\ref{chap:policy} examines policy and governance implications.
